\documentclass{article}
\usepackage{amsmath} 
\usepackage{amsfonts}
\usepackage{booktabs}
\usepackage[a4paper, margin=2.5cm]{geometry}
\usepackage{float}   % for [H]
\usepackage{graphicx}   % for \includegraphics
\usepackage{tabularx}
\usepackage[utf8]{inputenc}
\usepackage{geometry}
\usepackage{booktabs}
\usepackage{longtable}
\usepackage{blindtext}
\usepackage{hyperref}
\usepackage{natbib} % <-- NEW: to handle references
\usepackage{setspace}
\usepackage{array}
\usepackage{dcolumn}
\usepackage{threeparttable}
\usepackage{tikz}
\usepackage{amsmath}
\usetikzlibrary{decorations.pathreplacing}
\usepackage{pdflscape} % in your preamble
\usepackage{tabularray}
\setcounter{secnumdepth}{2}
\usepackage{amsmath, amsthm}  % for math and theorem environments
\setlength{\parskip}{0.45em}   % space between paragraphs
\setlength{\parindent}{0pt}    % optional: remove paragraph indentation

\usepackage{titlesec}

\titlespacing*{\section}
{0pt}      % left margin
{1.0em}    % space before section
{0.5em}    % space after section

\begin{document}

\title{Labor PS1: unemployment flows}
\author{Jordi Torres}
\date{\today}

\maketitle

\section{Intro}
I use monthly CPS gross flow data from the BLS for the period 1990--2019, focusing on men and women separately. The sample restriction ensures me the availability of seasonally adjusted employment, unemployment, and transition flow series. While Shimer (2012) begins in 1948 and uses short-term unemployment to approximate flows, I focus on the post-1990 period for simplicity and direct availability of gross flow data. Despite the shorter sample, the core replication exercise remains valid\footnote{Although Shimer (2012) constructs transition rates using short-term unemployment and other adjustments to extend the sample back to 1948, he notes that alternative measures differ mainly in level and have little effect on the cyclical fluctuations of the job-finding and separation rates. Such alternative constructions yield highly correlated series, supporting the use of CPS gross flows and continuous-time transformation in this exercise. \citep{Shimer2012}
}.

In Section~\ref{sec1}, I present and interpret the main results and compare them to Shimer (2012). Section~\ref{sec2} reports the corresponding tables and figures.


\section{Results}
\label{sec1}

Table~\ref{tab:rates} reports summary statistics for monthly job-finding ($F$) and separation probabilities ($S$) and their continuous-time hazard counterparts ($f, s$). Job-finding rates are substantially larger and more volatile than separation rates for both men and women. For example, the standard deviation of the job-finding hazard is around 0.06, while the separation hazard exhibits a standard deviation close to 0.002. This large difference indicates that hiring flows dominate separations in magnitude and variability.

Men exhibit higher average job-finding rates (0.311 vs 0.273 in hazard terms), whereas women display higher separation rates (0.033 vs 0.022). This suggests that male workers experience stronger hiring dynamics, while female workers exhibit slightly higher job separation intensity. These differences may reflect heterogeneity in sectoral composition, labor market attachment, or job stability across genders. Nevertheless, the qualitative ranking of flows is consistent with Shimer (2012), who documents that job-finding probabilities are much larger and more volatile than separation probabilities in aggregate U.S.\ data.

Figure~\ref{job_finding} shows the evolution of job-finding hazards over time, while Figure~\ref{job_separation} reports separation hazards. Job-finding rates display strong cyclical fluctuations, particularly during the Great Recession (2007–2009), where a sharp collapse is visible for both genders. In contrast, separation rates remain comparatively smooth and exhibit much smaller cyclical movements.

Figures~\ref{shimer_w} and \ref{shimer_m} present the Shimer decomposition for women and men. Table~\ref{tab:decomp} reports the standard deviation of three steady-state unemployment measures: (i) $u^{ss}$, the steady-state unemployment rate implied by the observed hazards; (ii) $u^{f}$, the counterfactual unemployment rate when separations are fixed at their sample mean and only job-finding varies; and (iii) $u^{s}$, the counterfactual unemployment rate when job-finding is fixed and only separations vary.

For women, the standard deviation of steady-state unemployment, $SD(u^{ss})$, is 0.021. When separations are held fixed and only job-finding varies, volatility remains high at 0.024. In contrast, when job-finding is held fixed and only separations vary, volatility drops sharply to 0.007. A similar pattern emerges for men. These results indicate that cyclical unemployment fluctuations are primarily driven by movements in job-finding rates rather than separations, which is one of the main conclusions of Shimer (2012).

Interestingly, the relative contribution of job-finding appears slightly stronger for women in this sample, as the gap between $SD(u_f)$ and $SD(u_s)$ is larger for women than for men. This suggests that female unemployment may be even more sensitive to hiring conditions.

Using both HP filtering and linear detrending of log series, I analyze the cyclical components of job-finding rates, separation rates, and unemployment. Tables~\ref{tab:sd_cycles} and \ref{tab:cor_cycles} report cyclical volatility and correlations. Under both detrending methods, job-finding rates are substantially more volatile than separation rates for both men and women. Moreover, job-finding is strongly negatively correlated with unemployment (between $-0.87$ and $-0.95$), whereas separation exhibits weaker and less stable correlations, even changing sign depending on the detrending method.

The qualitative conclusions are robust across detrending methods and are similar Shimer (2012): unemployment fluctuations are primarily driven by movements in job-finding rates rather than separations. Although the exact magnitudes differ—partly due to the shorter sample period (1990–2019) and the focus on gender subgroups—the central mechanism remains the same. Cyclical labor market adjustment operates mainly through hiring dynamics, even within demographic groups.




\section{Appendix}
\label{sec2}

\subsection{Exercise 2}
\label{ex1}


% Table created by stargazer v.5.2.3 by Marek Hlavac, Social Policy Institute. E-mail: marek.hlavac at gmail.com
% Date and time: Sun, Feb 15, 2026 - 23:04:33
\begin{table}[!htbp] \centering 
  \caption{Job Finding and Separation Rates (Monthly, 1990–2019)} 
  \label{tab:rates} 
\begin{tabular}{@{\extracolsep{5pt}} ccc} 
\\[-1.8ex]\hline 
\hline \\[-1.8ex] 
Statistic & Women & Men \\ 
\hline \\[-1.8ex] 
Mean F & $0.237$ & $0.265$ \\ 
SD F & $0.044$ & $0.046$ \\ 
Mean S & $0.032$ & $0.021$ \\ 
SD S & $0.002$ & $0.002$ \\ 
Mean f & $0.273$ & $0.310$ \\ 
SD f & $0.057$ & $0.062$ \\ 
Mean s & $0.033$ & $0.022$ \\ 
SD s & $0.002$ & $0.002$ \\ 
\hline \\[-1.8ex] 
\end{tabular} 
\end{table} 


\begin{figure}[H]
    \centering
    \includegraphics[width=0.8\textwidth]{job_finding_rate.png}
    \caption{Job-finding hazard over time}
    \label{job_finding}
\end{figure}

\begin{figure}[H]
    \centering
    \includegraphics[width=0.8\textwidth]{job_separation_rate.png}
    \caption{Job-separation hazard over time}
    \label{job_separation}

\end{figure}

\subsection{Exercise 3}
\label{ex2}

% Table created by stargazer v.5.2.3 by Marek Hlavac, Social Policy Institute. E-mail: marek.hlavac at gmail.com
% Date and time: Sun, Feb 15, 2026 - 11:10:06
\begin{table}[!htbp] \centering 
  \caption{Shimer Decomposition: Volatility of Steady-State Unemployment} 
  \label{tab:decomp} 
\begin{tabular}{@{\extracolsep{5pt}} cccc} 
\\[-1.8ex]\hline 
\hline \\[-1.8ex] 
Group & SD\_u\_ss & SD\_u\_f & SD\_u\_s \\ 
\hline \\[-1.8ex] 
Women & 0.021 & 0.024 & 0.007 \\ 
Men & 0.016 & 0.014 & 0.006 \\ 
\hline \\[-1.8ex] 
\end{tabular} 
\end{table} 

\noindent {\footnotesize \textit{Note:} $u_t^{ss}=s_t/(s_t+f_t)$. $u^f_t$ fixes separations; $u^s_t$ fixes job-finding.}


\begin{figure}[H]
    \centering
    \includegraphics[width=0.8\textwidth]{shimer_w.png}
    \caption{Shimer decomposition (women). 
    The steady-state unemployment rate is $u^{ss}_t = \frac{s_t}{s_t + f_t}$. 
    $u^f_t$ holds separations fixed at their sample mean, while $u^s_t$ holds job-finding fixed.}
    \label{shimer_w}
\end{figure}

\begin{figure}[H]
    \centering
    \includegraphics[width=0.8\textwidth]{shimer_m.png}
    \caption{Shimer decomposition (men). 
    Counterfactual series are constructed analogously to the women sample.}
    \label{shimer_m}
\end{figure}


\subsection{Exercise 4}
\label{ex3}


% Table created by stargazer v.5.2.3 by Marek Hlavac, Social Policy Institute. E-mail: marek.hlavac at gmail.com
% Date and time: Sun, Feb 15, 2026 - 11:10:07
\begin{table}[!htbp] \centering 
  \caption{Cyclical Volatility (SD of Detrended Log Series)} 
  \label{tab:sd_cycles} 
\begin{tabular}{@{\extracolsep{5pt}} ccccc} 
\\[-1.8ex]\hline 
\hline \\[-1.8ex] 
Method & Group & SD\_f & SD\_s & SD\_u \\ 
\hline \\[-1.8ex] 
HP & Women & $0.118$ & $0.055$ & $0.098$ \\ 
HP & Men & $0.111$ & $0.057$ & $0.106$ \\ 
Linear & Women & $0.225$ & $0.073$ & $0.173$ \\ 
Linear & Men & $0.202$ & $0.067$ & $0.168$ \\ 
\hline \\[-1.8ex] 
\end{tabular} 
\end{table} 

\noindent {\footnotesize \textit{Note:} SD$_f$, SD$_s$, and SD$_u$ denote the standard deviation of detrended log job-finding hazards ($f_t$), separation hazards ($s_t$), and steady-state unemployment ($u_t^{ss}$), respectively.}


% Table created by stargazer v.5.2.3 by Marek Hlavac, Social Policy Institute. E-mail: marek.hlavac at gmail.com
% Date and time: Sun, Feb 15, 2026 - 23:04:34
\begin{table}[!htbp] \centering 
  \caption{Cyclical Co-movement (Correlations with Unemployment Cycle)} 
  \label{tab:cor_cycles} 
\begin{tabular}{@{\extracolsep{5pt}} cccccc} 
\\[-1.8ex]\hline 
\hline \\[-1.8ex] 
Method & Group & Corr\_f\_u & Corr\_s\_u & Corr\_f\_u\_actual & Corr\_s\_u\_actual \\ 
\hline \\[-1.8ex] 
HP & Women & $$-$0.849$ & $0.162$ & $$-$0.746$ & $$-$0.358$ \\ 
HP & Men & $$-$0.839$ & $0.339$ & $$-$0.826$ & $$-$0.177$ \\ 
Linear & Women & $$-$0.941$ & $$-$0.245$ & $$-$0.935$ & $$-$0.575$ \\ 
Linear & Men & $$-$0.937$ & $$-$0.123$ & $$-$0.954$ & $$-$0.449$ \\ 
\hline \\[-1.8ex] 
\end{tabular} 
\end{table} 


\noindent {\footnotesize \textit{Note:} Corr$_{f,u}$ and Corr$_{s,u}$ denote correlations between detrended log job-finding or separation hazards and detrended log steady-state unemployment.}





\end{document}