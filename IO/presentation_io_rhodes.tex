\documentclass{beamer}

\usetheme{Copenhagen} % \!/ change this shit. 
\usecolortheme{dolphin} % Blue color theme

% Sidebar navigation similar to Frankfurt
\useoutertheme[subsection=false]{miniframes}

% Packages
\usepackage{graphicx, booktabs, amsmath, amssymb, hyperref, xcolor}
\usepackage{amsmath}
\usepackage{amssymb}
\usepackage{amsfonts}
\usepackage{graphicx}
% Title information
\title[RWZ (2021)]{Multiproduct intermediaries}
\author[Rhodes, Wanabe, Zhou]{JPE, 2021 \\Andrew Rhodes, Makoto Wanabe, Jidong Zhou}
\date{\today}

\begin{document}

% Title Slide
\begin{frame}
    \titlepage
\end{frame}

% Outline Slide
\begin{frame}{Outline}
    \tableofcontents
\end{frame}

% Section 1: Introduction
\section{Introduction}

\begin{frame}{Context and Tension}

\begin{itemize}
    \item Intermediaries play a central role in many markets:
    retailers, platforms, malls, app stores, TV platforms.

    \item They control product assortment and exclusivity,
    shaping access to consumers.

    \medskip

    \item \textbf{Tension:}
    DTC sales and disintermediation should weaken intermediaries.

    \item Yet intermediaries remain profitable,
    expand assortments, and rely increasingly on exclusivity.
    \item \textbf{ADD some data on context}
\end{itemize}

\end{frame}


\begin{frame}{Motivation}

\begin{itemize}
    \item Many intermediaries sell \emph{multiple products}
    and choose assortments strategically.

    \item Core question:
    how can a multiproduct intermediary create value
    and earn profits in the face of disintermediation? \textbf{wtf does this mean?}

    \medskip

    \item Standard theory:
    profits require lower prices or improved search.

    \item \textbf{This paper:}
    intermediaries can earn profits
    \emph{purely through assortment choice},
    even without lowering prices or search costs.

    \item Mechanism:
    assortment reallocates consumer search across products.
\end{itemize}

\end{frame}



\begin{frame}{Related Literature}

\begin{tabular}{p{0.6\textwidth} p{0.35\textwidth}}

\textbf{Intermediaries:}\\
search, certification, information\\
{\scriptsize\textcolor{gray}{(Rubinstein--Wolinsky 1987; Gehrig 1993; Spulber 1996)}}
&
\makebox[\linewidth][c]{\textit{Profits without improving search}}
\\[0.45cm]

\textbf{Bundling:}\\
pricing-based mechanisms\\
{\scriptsize\textcolor{gray}{(Stigler 1968; Adams--Yellen 1976; McAfee et al. 1989)}}
&
\makebox[\linewidth][c]{\textit{Assortment-based bundling}}
\\[0.45cm]

\textbf{Multiproduct search:}\\
exogenous product ranges\\
{\scriptsize\textcolor{gray}{(McAfee 1995; Shelegia 2012; Zhou 2014; Rhodes 2015)}}
&
\makebox[\linewidth][c]{\textit{Endogenous assortment choice}}

\end{tabular}

\end{frame}

\section{Model}

\begin{frame}{Setting}
\begin{itemize}
    \item \textbf{Products / Manufacturers}
    \begin{itemize}
        \item Continuum of products \(i\in[0,1]\), marginal cost \(c_i\ge0\)
        \item Demand \(Q_i(p)\), monopoly price \(p_i^m\)
        \item Per-consumer profit and surplus:
        \[
            \pi_i=(p_i^m-c_i)Q_i(p_i^m),\qquad
            v_i=\int_{p_i^m}^{\infty}Q_i(p)\,dp
        \]
    \end{itemize}

    \item \textbf{Consumers}
    \begin{itemize}
        \item Unit mass, additive utility across products
        \item Identical preferences; heterogeneity only in search cost \(s\sim F\)
        \item Observe availability, not prices
    \end{itemize}

    \item \textbf{Intermediary}
    \begin{itemize}
        \item Chooses assortment \(A\subset[0,1]\), \(|A|\le\bar m\)
        \item TIOLI contracts \((\tau_i,T_i)\), exclusive or not
        \item Search cost \(h(|A|)\cdot s\)
    \end{itemize}
\end{itemize}
\end{frame}




\begin{frame}{Lemma 1 (Pricing and Contracting)}
\begin{lemma}
In any equilibrium where product markets are active:
\begin{enumerate}
    \item All sellers of product \(i\) charge the monopoly price
    \(p_i = p_i^m\).

    \item If product \(i\) is stocked by the intermediary,
    there exists an equilibrium contract with
    \[
        \tau_i = c_i,
        \qquad
        T_i = \pi_i F(v_i),
    \]
    both under exclusivity and non-exclusivity.
\end{enumerate}
\end{lemma}


\medskip
\textbf{Implication:}
products can be indexed by
\[
(\pi_i, v_i)\in\mathbb{R}_+^2,
\]
with joint distribution \(G(\pi,v)\).
\end{frame}


\begin{frame}{Simple Case: Consumer Decision}

\textbf{Assumptions:}
exclusivity, \(h(m)=m\), \(\bar m=1\)

\medskip

Intermediary stocks \(A \subset \Omega\) exclusively.

\[
\text{Visit } I
\;\Longleftrightarrow\;
\underbrace{\int_A v\,dG}_{\text{expected surplus}}
\;\ge\;
\underbrace{s \int_A dG}_{\text{search cost}}
\]

\[
\Longleftrightarrow
\quad
s \le
\hat v
\equiv
\frac{\int_A v\,dG}{\int_A dG}
\]

\medskip

\centering
\textit{\scriptsize
Consumers compare average surplus to their search cost.
}

\end{frame}

\begin{frame}{Simple Case: Intermediary Problem}

Consumers visiting intermediary: \(F(\hat v)\)

\medskip

Net profit from product \((\pi,v)\):
\[
\pi\,[F(\hat v)-F(v)]
\]

\scriptsize
(gains from extra consumers
\(-\)
lump-sum paid to manufacturer)

\medskip
\textbf{define OMEGA!!!}
\[
\max_{A \subset \Omega}
\int_A \pi\,[F(\hat v)-F(v)]\, dG
\]

\medskip

\centering
\textit{\scriptsize
Low-\(v\) products earn profits.
High-\(v\) products attract consumers.
}

\end{frame}





\begin{frame}{Solution: Optimal Product Selection}

\textbf{Reformulation}

Stocking decision:
\[
q(\pi,v)=1 \;\Longleftrightarrow\; (\pi,v)\in A
\]

\medskip

\textbf{Intermediary problem}
\[
\max_q \int_{\Omega} q(\pi,v)\,
\Big[
\underbrace{\pi(F(\hat v)-F(v))}_{\text{direct profit}}
+
\underbrace{\lambda (v-\hat v)}_{\text{search externality}}
\Big] dG
\]

\medskip


\end{frame}

\begin{frame}{Solution: Optimal Product Selection}
\textcolor{blue}{\textbf{mention proposition 1?}}
    \begin{figure}[H]
    \centering
    \includegraphics[width=0.9\textwidth]{io_presentation_pic1.png}
\end{figure}  

\centering
\textit{\scriptsize
High-\(\pi\), low-\(v\) products make money.
Low-\(\pi\), high-\(v\) products attract consumers.
}
\end{frame}



\section{Discussion}

\begin{frame}{Beyond the Simple Model}

\begin{itemize}
    \item The paper extends the benchmark to a richer environment:
    \begin{itemize}
        \item endogenous exclusivity vs non-exclusivity
        \item capacity constraint on assortment size ($\bar m$)
        \item general search cost technology $h(m)$
    \end{itemize}

    \medskip

    \item \textbf{Key insight:}
    the same search-reallocation mechanism survives,
    but now interacts with:
    \begin{itemize}
        \item the exclusivity margin
        \item the capacity margin
    \end{itemize}

    \medskip

    \item As a result, the intermediary may optimally stock:
    \begin{itemize}
        \item traffic-generating products
        \item profit-generating products
        \item \emph{and} products that do both.
    \end{itemize}
\end{itemize}

\end{frame}

\begin{frame}{Applications (Overview)}

\begin{itemize}
    \item \textbf{Retail platforms / malls}
    \begin{itemize}
        \item Some products are used to \emph{attract consumers} (anchors),
        others to \emph{generate profits}.
        \item Explains subsidies to high-value sellers and cross-store externalities.
    \end{itemize}

    \medskip

    \item \textbf{Exclusivity and private labels}
    \begin{itemize}
        \item Exclusive products optimally have high consumer surplus
        but low standalone profitability.
        \item Predicts overuse of exclusivity relative to the social optimum.
    \end{itemize}

    \medskip

    \item \textbf{Direct-to-consumer (DTC) sales}
    \begin{itemize}
        \item Easier DTC weakens the intermediary and shrinks its assortment.
        \item Intermediaries respond by relying more on exclusivity.
    \end{itemize}
\end{itemize}

\medskip
\centering
\textit{\scriptsize
Applications illustrate the mechanism; the contribution is the framework.
}
\end{frame}


\begin{frame}{Summary}

\begin{itemize}
    \item Introduces a new framework to study multiproduct intermediaries
    with consumer search frictions and endogenous assortment.

    \medskip

    \item Main result:
    a multiproduct intermediary can earn strictly positive profits
    \emph{without improving prices or search efficiency}.

    \medskip

    \item Mechanism:
    assortment choice reallocates consumer search
    across products with different roles.

    \medskip

    \item The framework provides a unified way to think about
    exclusivity, capacity, and DTC competition.
\end{itemize}

\end{frame}

\begin{frame}{Discussion and Limitations}

\begin{itemize}
    \item \textbf{Scope of the pricing environment}
    \begin{itemize}
        \item Prices are fixed at monopoly levels to isolate pure assortment and exclusivity effects.
        \item The paper establishes profitability under this austere benchmark, but does not quantify how strong these forces are relative to pricing distortions.
    \end{itemize}

    \medskip

    \item \textbf{Absence of intermediary competition}
    \begin{itemize}
        \item The analysis focuses on a single intermediary.
        \item With competing intermediaries, exclusivity may be disciplined by consumer switching and differentiation, potentially altering welfare conclusions without eliminating profitability.
    \end{itemize}

    \medskip

    \item \textbf{Consumer search structure}
    \begin{itemize}
        \item Consumer heterogeneity operates solely through search costs.
        \item The sensitivity of the mechanism to alternative forms of consumer search or platform choice is not fully characterized.
    \end{itemize}
\end{itemize}

\end{frame}


\end{document}