\documentclass{beamer}

\usetheme{AnnArbor} % \!/ change this shit. 
\usecolortheme{dolphin} % Blue color theme

% Sidebar navigation similar to Frankfurt
\useoutertheme[subsection=false]{miniframes}

% Packages
\usepackage{graphicx, booktabs, amsmath, amssymb, hyperref, xcolor}
\usepackage{amsmath}
\usepackage{amssymb}
\usepackage{amsfonts}
\usepackage{graphicx}
% Title information
\title[RWZ (2021)]{Multiproduct intermediaries}
\author[Rhodes, Wanabe, Zhou]{JPE, 2021 \\Andrew Rhodes, Makoto Wanabe, Jidong Zhou}
\date{\today}

\begin{document}

% Title Slide
\begin{frame}
    \titlepage
\end{frame}

% Outline Slide
\begin{frame}{Outline}
    \tableofcontents
\end{frame}

% Section 1: Introduction
\section{Introduction}


\begin{frame}{Motivation}
\begin{itemize}
    \item Many intermediaries sell \emph{multiple products}
    (retailers, malls, platforms, TV platforms).

    \item A core strategic decision:
    product assortment and exclusivity.

    \item Two forces reshape this decision:
    direct-to-consumer (DTC) sales and exclusivity agreements.

    \item Existing theory offers little guidance
    on optimal assortment choices.
\end{itemize}
\end{frame}


\begin{frame}{Motivation}
\begin{itemize}
    \item \textbf{Question:}
    how can a multiproduct intermediary create value and earn profits?

    \item Standard models:
    profits require lower prices or better search.

    \item \textbf{Key result:}
    a multiproduct intermediary can earn strictly positive profits
    \emph{without reducing prices or search costs}.

    \item Profits arise from assortment choice
    and cross-product search incentives.
\end{itemize}
\end{frame}

\section{Model}
\begin{frame}{Related Literature}
\small

\textbf{Intermediaries}
\begin{itemize}\setlength{\itemsep}{1pt}
    \item Search / matching efficiency 
    {\footnotesize (Rubinstein--Wolinsky, 1987; Gehrig, 1993; Spulber, 1996)}
    \item Certification, asymmetric information 
    {\footnotesize (Biglaiser, 1993; Lizzeri, 1999)}
\end{itemize}

{\scriptsize $\Rightarrow$ \textbf{This paper:}}
\small intermediaries earn profits even without improving search efficiency.

\vspace{0.2cm}

\textbf{Bundling}
\begin{itemize}\setlength{\itemsep}{1pt}
    \item Classic bundling mechanisms 
    {\footnotesize (Stigler, 1968; Adams--Yellen, 1976; McAfee et al., 1989)}
\end{itemize}

{\scriptsize $\Rightarrow$ \textbf{This paper:}}
\small focuses on which products are bundled via endogenous assortment choice.

\vspace{0.2cm}

\textbf{Multiproduct search}
\begin{itemize}\setlength{\itemsep}{1pt}
    \item Consumer search with exogenous product ranges 
    {\footnotesize (McAfee, 1995; Shelegia, 2012; Zhou, 2014; Rhodes, 2015)}
\end{itemize}

{\scriptsize $\Rightarrow$ \textbf{This paper:}}
\small endogenous product range choice with upstream manufacturers and vertical structure.

\end{frame}


\begin{frame}{Setting}
\begin{itemize}
    \item \textbf{Products / Manufacturers}
    \begin{itemize}
        \item Continuum of products \(i\in[0,1]\), marginal cost \(c_i\ge0\)
        \item Demand \(Q_i(p)\), monopoly price \(p_i^m\)
        \item Per-consumer profit and surplus:
        \[
            \pi_i=(p_i^m-c_i)Q_i(p_i^m),\qquad
            v_i=\int_{p_i^m}^{\infty}Q_i(p)\,dp
        \]
    \end{itemize}

    \item \textbf{Consumers}
    \begin{itemize}
        \item Unit mass, additive utility across products
        \item Identical preferences; heterogeneity only in search cost \(s\sim F\)
        \item Observe availability, not prices
    \end{itemize}

    \item \textbf{Intermediary}
    \begin{itemize}
        \item Chooses assortment \(A\subset[0,1]\), \(|A|\le\bar m\)
        \item TIOLI contracts \((\tau_i,T_i)\), exclusive or not
        \item Search cost \(h(|A|)\cdot s\)
    \end{itemize}
\end{itemize}
\end{frame}



\begin{frame}{Timing}

    \textcolor{blue}{\textbf{Add a simple line, that is all, shimplesh}}

\begin{enumerate}
    \item The intermediary simultaneously makes TIOLI offers
    \((\tau_i,T_i)\) to manufacturers,
    specifying exclusivity or non-exclusivity.
    Manufacturers accept or reject.

    \item All firms that sell to consumers
    set retail prices for their products.

    \item Consumers observe availability,
    form rational expectations over prices,
    then search sequentially and purchase.
\end{enumerate}
\end{frame}

\begin{frame}{Lemma 1 (Pricing and Contracting)}
\begin{lemma}
In any equilibrium where product markets are active:
\begin{enumerate}
    \item All sellers of product \(i\) charge the monopoly price
    \(p_i = p_i^m\).

    \item If product \(i\) is stocked by the intermediary,
    there exists an equilibrium contract with
    \[
        \tau_i = c_i,
        \qquad
        T_i = \pi_i F(v_i),
    \]
    both under exclusivity and non-exclusivity.
\end{enumerate}
\end{lemma}


\textcolor{blue}{\textbf{Add definition of $\Omega$!!!!!!!!}}

\medskip
\textbf{Implication:}
products can be indexed by
\[
(\pi_i, v_i)\in\mathbb{R}_+^2,
\]
with joint distribution \(G(\pi,v)\).
\end{frame}


\begin{frame}{Simple Case: Consumer Decision}

\textbf{Assumptions:}
exclusivity, \(h(m)=m\), \(\bar m=1\)

\medskip

Intermediary stocks \(A \subset \Omega\) exclusively.

\[
\text{Visit } I
\;\Longleftrightarrow\;
\underbrace{\int_A v\,dG}_{\text{expected surplus}}
\;\ge\;
\underbrace{s \int_A dG}_{\text{search cost}}
\]

\[
\Longleftrightarrow
\quad
s \le
\hat v
\equiv
\frac{\int_A v\,dG}{\int_A dG}
\]

\medskip

\centering
\textit{\scriptsize
Consumers compare average surplus to their search cost.
}

\end{frame}

\begin{frame}{Simple Case: Intermediary Problem}

Consumers visiting intermediary: \(F(\hat v)\)

\medskip

Net profit from product \((\pi,v)\):
\[
\pi\,[F(\hat v)-F(v)]
\]

\scriptsize
(gains from extra consumers
\(-\)
lump-sum paid to manufacturer)

\medskip

\[
\max_{A \subset \Omega}
\int_A \pi\,[F(\hat v)-F(v)]\, dG
\]

\medskip

\centering
\textit{\scriptsize
Low-\(v\) products earn profits.
High-\(v\) products attract consumers.
}

\end{frame}





\begin{frame}{Solution: Optimal Product Selection}

\textbf{Reformulation}

Stocking decision:
\[
q(\pi,v)=1 \;\Longleftrightarrow\; (\pi,v)\in A
\]

\medskip

\textbf{Intermediary problem}
\[
\max_q \int_{\Omega} q(\pi,v)\,
\Big[
\underbrace{\pi(F(\hat v)-F(v))}_{\text{direct profit}}
+
\underbrace{\lambda (v-\hat v)}_{\text{search externality}}
\Big] dG
\]

\medskip

\textbf{Optimal policy: cutoff structure}

\[
q(\pi,v)=1 \;\Longleftrightarrow\;
\begin{cases}
v<\hat v \;\;\text{and}\;\;
\pi \ge \dfrac{\hat v-v}{F(\hat v)-F(v)} \\
v>\hat v \;\;\text{and}\;\;
\pi \le \dfrac{\hat v-v}{F(\hat v)-F(v)}
\end{cases}
\]

\medskip

\centering
\textit{\scriptsize
High-\(\pi\), low-\(v\) products make money.
Low-\(\pi\), high-\(v\) products attract consumers.
}

\textcolor{blue}{\textbf{Clarify the constraint, where does it come from}}

\end{frame}

\begin{frame}{Solution: Optimal Product Selection}
\textcolor{blue}{\textbf{mention proposition 1?}}
    \begin{figure}[H]
    \centering
    \includegraphics[width=0.9\textwidth]{io_presentation_pic1.png}
\end{figure}    
\end{frame}



\section{Discussion}

\begin{frame}{Main Model}
\begin{itemize}
    \item richer environment
    \begin{enumerate}
        \item \(q(\pi, v)\in \left((0,0), (1,0), (0,1)\right)\), allow for $(q^{e}, q^{ne})$ 
        \item \(\hat{m}\leq 1\)
        \item \(h(m)\lesseqqgtr m\)
    \end{enumerate}
    \item $\rightarrow$ adds one more layer of complexity: firms also find it optimal to  provide high $\pi$, high $v$ goods. 
    \item Now results endogeneously depend on $h(m), \bar{m}$, this provides scope for richer applications: exclusivity/nonexclusivity margin and caapcity margin. 
    \item \textbf{Applications}: re
\end{itemize}
\end{frame}


\begin{frame}{Summary}
\begin{itemize}
\item New framework to study multiproduct intermediaries when consumers face heterogeneous demand and search frictions. 
\item Main result: show that a multiproduct intermediate is profitable even if it is does not provide search efficiencies.
\item Mechanism: bundling + effects of technology/exclusivity under more complex contracts
\item Provide applications to DTCs. 
\end{itemize}
\end{frame}


\begin{frame}{Critiques/review}
\begin{itemize}
\item I would add critique + possible solutions; be nice and fair with them. This is to do. 
\item Mostly, consumers are passive agents, accept the monopoly price.
\item Also, absence of price competition among I (only one; this is the most obvious critique, but their model serves the purpose of starting a conversation)->they aknowledge this... and actually is part of their research agenda. How would this look like? 
\item What else?
\end{itemize}
\end{frame}



\end{document}