\documentclass{beamer}

\usetheme{Copenhagen} % \!/ change this shit. 
\usecolortheme{dolphin} % Blue color theme

% Sidebar navigation similar to Frankfurt
\useoutertheme[subsection=false]{miniframes}

% Packages
\usepackage{graphicx, booktabs, amsmath, amssymb, hyperref, xcolor}
\usepackage{amsmath}
\usepackage{amssymb}
\usepackage{amsfonts}
\usepackage{graphicx}
% Title information
\title[RWZ (2021)]{Multiproduct intermediaries}
\author[Rhodes, Watanabe, Zhou]{JPE, 2021 \\Andrew Rhodes, Makoto Watanabe, Jidong Zhou}
\date{\today}

\begin{document}

% Title Slide
\begin{frame}
    \titlepage
\end{frame}


% Section 1: Introduction
\section{Introduction}

\begin{frame}{Market Trends: An Empirical Tension}

\begin{itemize}

\item \textbf{Direct-to-Consumer (DTC) sales are expanding:}
\begin{itemize}
    \item European Commission (2017):
    In several EU sectors, more than 50\% of manufacturers sell directly online (e.g clothing 85\%)
    \item OECD (2019): E-commerce has facilitated manufacturers' expansion
    into direct retail channels.
\end{itemize}

\medskip

\item \textbf{Yet multiproduct intermediaries remain dominant:}
\begin{itemize}
    \item Amazon net sales exceed \$500bn annually (Amazon 10-K).
    \item Growth of private labels and exclusive products  (e.g., streaming platforms securing exclusive sports and movie rights).

    \item Large retailers continue to operate broad assortments.
\end{itemize}

\medskip

\item \textbf{Puzzle:}
If manufacturers can reach consumers directly,
why do multiproduct intermediaries remain profitable
and rely increasingly on exclusivity?

\end{itemize}
\end{frame}



\begin{frame}{Motivation}

\begin{itemize}
    \item Many intermediaries sell \emph{multiple products}
    and choose assortments strategically.

    \item Core question:
    How can a multiproduct intermediary remain profitable when manufacturers can sell directly and it does not lower prices?

    \medskip

    \item Standard models of intermediation:
    profits arise from
    \begin{itemize}
        \item reducing search frictions, or
        \item lowering prices .
    \end{itemize}


    \item \textbf{This paper:}
    intermediaries can earn profits
    \emph{purely through assortment choice},
    even without lowering prices or search costs.

    \item Mechanism:
    assortment reallocates consumer search across products.
\end{itemize}

\end{frame}


\begin{frame}{Related Literature}

\begin{itemize}

\item \textbf{Intermediation models}
\begin{itemize}
    \item Search, certification, information {\scriptsize (Rubinstein--Wolinsky 1987; Gehrig 1993; Spulber 1996)}
\end{itemize}
{\color{blue} $\Rightarrow$ Profits without reducing search frictions}

\medskip

\item \textbf{Bundling}
\begin{itemize}
    \item Pricing-based mechanisms {\scriptsize (Stigler 1968; Adams--Yellen 1976; McAfee et al. 1989)}
\end{itemize}
{\color{blue} $\Rightarrow$ Assortment-based bundling}

\medskip

\item \textbf{Multiproduct search}
\begin{itemize}
    \item Exogenous product ranges {\scriptsize (McAfee 1995; Shelegia 2012; Zhou 2014; Rhodes 2015)}
\end{itemize}
{\color{blue} $\Rightarrow$ Endogenous assortment choice}

\end{itemize}

\end{frame}



\section{Model}

\begin{frame}{Setting}
\begin{itemize}
    \item \textbf{Products / Manufacturers}
    \begin{itemize}
        \item Continuum of products \(i\in[0,1]\), marginal cost \(c_i\ge0\)
        \item Demand \(Q_i(p)\), monopoly price \(p_i^m\)
        \item Per-consumer profit and surplus:
        \[
            \pi_i=(p_i^m-c_i)Q_i(p_i^m),\qquad
            v_i=\int_{p_i^m}^{\infty}Q_i(p)\,dp
        \]
    \end{itemize}

    \item \textbf{Consumers}
    \begin{itemize}
        \item Unit mass, additive utility across products
        \item Identical preferences; heterogeneity only in search cost \(s\sim F\)
        \item Observe availability, not prices
    \end{itemize}

    \item \textbf{Intermediary}
    \begin{itemize}
        \item Chooses assortment \(A\subset[0,1]\), \(|A|\le\bar m\)
        \item TIOLI contracts \((\tau_i,T_i)\), exclusive or not
        \item Search cost \(h(|A|)\cdot s\)
    \end{itemize}
\end{itemize}
\end{frame}



\begin{frame}{Pricing and Contracting}

\begin{itemize}
    \item In equilibrium, all sellers charge monopoly prices $p_i^m$.
    \item Two-part contracts allow the intermediary to extract manufacturer surplus.
\end{itemize}

\medskip

\[
\Rightarrow \text{Products can be indexed by } (\pi_i, v_i)
\]

\medskip

Let $G(\pi,v)$ denote the joint distribution with support $\Omega$.

\end{frame}




\begin{frame}{Simple Case: Consumer Decision}

\textbf{Assumptions:}
exclusivity, \(h(m)=m\), \(\bar m=1\)

\medskip

Intermediary stocks \(A \subset \Omega\) exclusively.

\[
\text{Visit } I
\;\Longleftrightarrow\;
\underbrace{\int_A v\,dG}_{\text{expected surplus}}
\;\ge\;
\underbrace{s \int_A dG}_{\text{search cost}}
\]

\[
\Longleftrightarrow
\quad
s \le
\hat v
\equiv
\frac{\int_A v\,dG}{\int_A dG}
\]

\medskip

\centering
\textit{\scriptsize
Consumers compare average surplus to their search cost.
}

\end{frame}

\begin{frame}{Simple Case: Intermediary Problem}

Consumers visiting intermediary: \(F(\hat v)\)

\medskip

Net profit from product \((\pi,v)\):
\[
\pi\,[F(\hat v)-F(v)]
\]

\scriptsize
(gains from extra consumers
\(-\)
lump-sum paid to manufacturer)

\medskip
\textbf{define OMEGA!!!}
\[
\max_{A \subset \Omega}
\int_A \pi\,[F(\hat v)-F(v)]\, dG
\]

\medskip

\centering
\textit{\scriptsize
Low-\(v\) products earn profits.
High-\(v\) products attract consumers.
}

\end{frame}





\begin{frame}{Solution: Optimal Product Selection}

\textbf{Reformulation}

Stocking decision:
\[
q(\pi,v)=1 \;\Longleftrightarrow\; (\pi,v)\in A
\]

\medskip

\textbf{Intermediary problem}
\[
\max_q \int_{\Omega} q(\pi,v)\,
\Big[
\underbrace{\pi(F(\hat v)-F(v))}_{\text{direct profit}}
+
\underbrace{\lambda (v-\hat v)}_{\text{search externality}}
\Big] dG
\]

\centering
\textit{\scriptsize
where $\lambda$ is the multiplier capturing the marginal value of attracting consumers.
}
\medskip


\end{frame}

\begin{frame}{Solution: Optimal Product Selection}
    \begin{figure}[H]
    \centering
    \includegraphics[width=0.9\textwidth]{io_presentation_pic1.png}
\end{figure}  

\centering
\textit{\scriptsize
High-\(\pi\), low-\(v\) products make money.
Low-\(\pi\), high-\(v\) products attract consumers.
}
\end{frame}



\section{Discussion}

\begin{frame}{Beyond the Simple Model}

\begin{itemize}
    \item The paper extends the benchmark to a richer environment:
    \begin{itemize}
        \item endogenous exclusivity vs non-exclusivity
        \item capacity constraint on assortment size ($\bar m$)
        \item general search cost technology $h(m)$
    \end{itemize}

    \medskip

    \item \textbf{Key insight:}
    the same search-reallocation mechanism survives,
    but now interacts with:
    \begin{itemize}
        \item the exclusivity margin
        \item the capacity margin
    \end{itemize}

    \medskip

    \item As a result, the intermediary may optimally stock:
    \begin{itemize}
        \item traffic-generating products
        \item profit-generating products
        \item \emph{and} products that do both.
    \end{itemize}
\end{itemize}

\end{frame}

\begin{frame}{Applications (Overview)}

\begin{itemize}
    \item \textbf{Retail platforms / malls}
    \begin{itemize}
        \item Some products are used to \emph{attract consumers} (anchors),
        others to \emph{generate profits}.
        \item Explains subsidies to high-value sellers and cross-store externalities.
    \end{itemize}

    \medskip

    \item \textbf{Exclusivity and private labels}
    \begin{itemize}
        \item Exclusive products optimally have high consumer surplus
        but low standalone profitability.
        \item Predicts overuse of exclusivity relative to the social optimum.
    \end{itemize}

    \medskip

    \item \textbf{Direct-to-consumer (DTC) sales}
    \begin{itemize}
        \item Easier DTC weakens the intermediary and shrinks its assortment.
        \item Intermediaries respond by relying more on exclusivity.
    \end{itemize}
\end{itemize}

\medskip

\end{frame}


\begin{frame}{Summary}

\begin{itemize}
    \item Introduces a new framework to study multiproduct intermediaries
    with consumer search frictions and endogenous assortment.

    \medskip

    \item Main result:
    a multiproduct intermediary can earn strictly positive profits
    \emph{without improving prices or search efficiency}.

    \medskip

    \item Mechanism:
    assortment choice reallocates consumer search
    across products with different roles.

    \medskip

    \item The framework provides a unified way to think about
    exclusivity, capacity, and DTC competition.
\end{itemize}

\end{frame}

\begin{frame}{Discussion and Limitations}

\begin{itemize}
\item \textbf{Pricing environment}
\begin{itemize}
    \item Retail prices are fixed at monopoly levels.
    \item The intermediary cannot adjust prices or eliminate double marginalization.
    \item In practice, pricing and assortment decisions interact.
\end{itemize}


    \medskip

\item \textbf{No intermediary competition}
\begin{itemize}
    \item The model considers a single intermediary.
    \item With competing intermediaries, platforms may compete for traffic and exclusive products.
    \item Competition could reduce the profitability of distortionary assortment choices.
\end{itemize}


    \medskip

    \item \textbf{Consumer search structure}
    \begin{itemize}
        \item Consumer heterogeneity operates only through search costs.
        \item The sensitivity of the mechanism to alternative forms of consumer search or platform choice is not fully characterized.
    \end{itemize}
\end{itemize}

\end{frame}


\end{document}