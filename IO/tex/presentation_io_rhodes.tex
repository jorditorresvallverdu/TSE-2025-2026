\documentclass{beamer}

\usetheme{Copenhagen} % \!/ change this shit. 
\usecolortheme{dolphin} % Blue color theme

% Sidebar navigation similar to Frankfurt
\useoutertheme[subsection=false]{miniframes}

% Packages
\usepackage{graphicx, booktabs, amsmath, amssymb, hyperref, xcolor}
\usepackage{amsmath}
\usepackage{amssymb}
\usepackage{amsfonts}
\usepackage{graphicx}
% Title information
\title[RWZ (2021)]{Multiproduct intermediaries}
\author[Rhodes, Watanabe, Zhou]{JPE, 2021 \\Andrew Rhodes, Makoto Watanabe, Jidong Zhou}
\date{\today}

\begin{document}

% Title Slide
\begin{frame}
    \titlepage
\end{frame}


% Section 1: Introduction
\section{Introduction}

\begin{frame}{Market Trends: An Empirical Tension}

\begin{itemize}

\item \textbf{Direct-to-Consumer (DTC) sales are expanding:}
\begin{itemize}
    \item European Commission (2017):
    In several EU sectors, more than 50\% of manufacturers sell directly online (e.g clothing 85\%)
    \item OECD (2019): E-commerce has facilitated manufacturers' expansion
    into direct retail channels.
\end{itemize}

\medskip

\item \textbf{Yet multiproduct intermediaries remain dominant:}
\begin{itemize}
    \item Amazon net sales exceed \$500bn annually (Amazon 10-K).
    \item Growth of private labels and exclusive products  (e.g., streaming platforms securing exclusive sports and movie rights).

    \item Large retailers continue to operate broad assortments.
\end{itemize}

\medskip

\item \textbf{Puzzle:}
If manufacturers can reach consumers directly,
why do multiproduct intermediaries remain profitable
and rely increasingly on exclusivity?

\end{itemize}
\end{frame}



\begin{frame}{Motivation}

\begin{itemize}
    \item Many intermediaries sell \emph{multiple products}
    and choose assortments strategically.

    \item Core question:
    How can a multiproduct intermediary remain profitable when manufacturers can sell directly and it does not lower prices?

    \medskip

    \item Standard models of intermediation:
    profits arise from
    \begin{itemize}
        \item reducing search frictions, or
        \item lowering prices .
    \end{itemize}


    \item \textbf{This paper:}
    intermediaries can earn profits
    \emph{purely through assortment choice},
    even without lowering prices or search costs.

    \item Mechanism:
    assortment reallocates consumer search across products.
\end{itemize}

\end{frame}


\begin{frame}{Related Literature}

\begin{itemize}

\item \textbf{Intermediation models}
\begin{itemize}
    \item Search, certification, information {\scriptsize (Rubinstein--Wolinsky 1987; Gehrig 1993; Spulber 1996)}
\end{itemize}
{\color{blue} $\Rightarrow$ Profits without reducing search frictions}

\medskip

\item \textbf{Bundling}
\begin{itemize}
    \item Chore mechanisms {\scriptsize (Stigler 1968; Adams--Yellen 1976; McAfee et al. 1989)}
\end{itemize}
{\color{blue} $\Rightarrow$ Assortment-based bundling}

\medskip

\item \textbf{Multiproduct search}
\begin{itemize}
    \item Exogenous product ranges/price decisions {\scriptsize (McAfee 1995; Shelegia 2012; Zhou 2014; Rhodes 2015)}
\end{itemize}
{\color{blue} $\Rightarrow$ Endogenous assortment choice}

\end{itemize}

\end{frame}



\section{Model}

\begin{frame}{Setting}
\begin{itemize}
    \item \textbf{Products / Manufacturers}
    \begin{itemize}
        \item Continuum of products \(i\in[0,1]\), marginal cost \(c_i\ge0\)
        \item Per-consumer profit and surplus:
        \[
            \pi_i=(p_i^m-c_i)Q_i(p_i^m),\qquad
            v_i=\int_{p_i^m}^{\infty}Q_i(p)\,dp
        \]
    \end{itemize}

    \item \textbf{Consumers}
    \begin{itemize}
        \item Unit mass, additive utility across products
        \item Identical preferences;
        \item  Pay a search cost \(s\sim F\)
    \end{itemize}

\item \textbf{Intermediary}
\begin{itemize}
    \item Chooses assortment \(A \subset [0,1]\), with \(m = |A|\)
    \item Capacity constraint: \(m \le \bar m < 1\)
    \item Search efficiencies \(h(m)\)
\end{itemize}

\end{itemize}
\end{frame}


\begin{frame}{Timing}

\[
\textbf{1. Contracts (TIOLI)}
\;\longrightarrow\;
\textbf{2. Pricing}
\;\longrightarrow\;
\textbf{3. Search \& Purchase}
\]

\vspace{0.5cm}

\begin{itemize}
    \item \textbf{1.} Intermediary offers $(\tau_i, T_i,$ exclusivity$)$
    \item \textbf{2.} Active sellers choose retail prices
    \item \textbf{3.} Consumers observe availability and search
\end{itemize}

\end{frame}




\begin{frame}{Pricing and Contracting: Key Implications}

\begin{itemize}
    \item Consumers do not observe prices before search
    \item $\Rightarrow$ Each seller charges monopoly price $p_i^m$
\end{itemize}

\medskip


Exclusive contract:
\[
\tau_i = c_i,
\qquad
T_i = \pi_i F(v_i)
\]

\medskip

\[
\Rightarrow \text{Each product is summarized by } (\pi_i, v_i)
\]


\end{frame}





\begin{frame}{Simple Case: Consumer Decision}

Let $\Omega \subset \mathbb{R}_+^2$ denote the set of feasible
$(\pi,v)$ pairs, with distribution $G(\pi,v)$.

\medskip

\textbf{Assumptions:}
exclusivity, \(h(m)=m\), \(\bar m=1\)
\medskip

\[
\text{Visit } I
\;\Longleftrightarrow\;
\underbrace{\int_A v\,dG}_{\text{expected surplus}}
\;\ge\;
\underbrace{s \int_A dG}_{\text{search cost}}
\]

\[
\Longleftrightarrow
\quad
s \le
\hat v
\equiv
\frac{\int_A v\,dG}{\int_A dG}
\]

\medskip

\centering
\textit{\scriptsize
Consumers compare average surplus to their search cost.
}

\end{frame}

\begin{frame}{Simple Case: Intermediary Problem}

Consumers visiting intermediary: \(F(\hat v)\)

\medskip

Net profit from product \((\pi,v)\):
\[
\pi\,[F(\hat v)-F(v)]
\]

\scriptsize
(gains from extra consumers
\(-\)
lump-sum paid to manufacturer)

\medskip
\[
\max_{A \subset \Omega}
\int_A \pi\,[F(\hat v)-F(v)]\, dG
\]

\medskip

\centering
\textit{\scriptsize
Low-\(v\) products earn profits.
High-\(v\) products attract consumers.
}

\end{frame}





\begin{frame}{Solution: Optimal Product Selection}

\textbf{Reformulation}

Stocking decision:
\[
q(\pi,v)=1 \;\Longleftrightarrow\; (\pi,v)\in A
\]

\medskip

\textbf{Intermediary problem}

\[
\max_q \int_{\Omega} q(\pi,v)\,
\pi(F(\hat v)-F(v))\, dG
\quad
\text{s.t. }
\hat v = 
\frac{\int_{\Omega} q(\pi,v)\, v\, dG}
{\int_{\Omega} q(\pi,v)\, dG}
\]

\medskip

\[
\max_q \int_{\Omega} q(\pi,v)\,
\Big[
\underbrace{\pi(F(\hat v)-F(v))}_{\text{direct profit}}
+
\underbrace{\lambda (v-\hat v)}_{\text{search externality}}
\Big] dG
\]

\centering
\textit{\scriptsize
where $\lambda$ is the multiplier capturing the marginal value of attracting consumers.
}
\medskip

\end{frame}




\begin{frame}{Solution: Optimal Product Selection}
    \begin{figure}[H]
    \centering
    \includegraphics[width=0.9\textwidth]{io_presentation_pic1.png}
\end{figure}  

\centering
\textit{\scriptsize
High-\(\pi\), low-\(v\) products make money.
Low-\(\pi\), high-\(v\) products attract consumers.
}
\end{frame}



\section{Discussion}

\begin{frame}{Beyond the Simple Model}

\textbf{Three extensions:}

\begin{itemize}
    \item Endogenous exclusivity vs. non-exclusivity
    \item Capacity constraint on assortment ($\bar m$)
    \item General search technology $h(m)$
\end{itemize}

\medskip

\textbf{What survives:}

\begin{itemize}
    \item Same traffic–profit tradeoff
    \item High-$v$ products attract consumers
    \item Low-$v$ products generate profit
\end{itemize}

\medskip

\textbf{What changes:}

\begin{itemize}
    \item Search technology affects how much exclusivity is needed
    \item Larger capacity $\Rightarrow$ larger platform, fewer exclusive products
    \item With strong economies of search, exclusivity becomes less important
\end{itemize}

\medskip


\end{frame}



\begin{frame}{Applications}

\textbf{Shopping malls}

\begin{itemize}
    \item Mall acts as a platform (does not set prices).
    \item Same $\left(\pi,v\right)$ logic applies.
    \item High-$v$ stores (``anchor stores'') attract consumers.
    \item Mall may subsidize anchors to extract higher rents from other stores.
\end{itemize}

\medskip

\textbf{Direct-to-Consumer (DTC) expansion}

\begin{itemize}
    \item Easier DTC raises manufacturers’ outside option ($\pi F(v/\theta)$).
    \item Intermediary profits decline.
    \item Optimal response:
    \begin{itemize}
        \item Smaller assortment
        \item Higher fraction of exclusive products
        \item More polarized product range
    \end{itemize}
\end{itemize}

\end{frame}


\begin{frame}{Summary}

\begin{itemize}
    \item Multiproduct intermediaries can earn positive profits
    even without lowering prices or reducing search frictions.

    \medskip

    \item Mechanism:
    assortment reallocates consumer search.
    \begin{itemize}
        \item High-$v$ products attract consumers.
        \item Low-$v$ products generate profit.
    \end{itemize}

    \medskip

    \item The framework explains:
    \begin{itemize}
        \item exclusivity and anchor stores,
        \item capacity choices,
        \item and the impact of DTC expansion.
    \end{itemize}

\end{itemize}

\end{frame}



\begin{frame}{Discussion and Limitations}

\begin{itemize}

\item \textbf{Pricing}
\begin{itemize}
    \item Prices remain at monopoly levels (no price competition).
    \item Assortment, not pricing, drives profits.
\end{itemize}

\medskip

\item \textbf{Market structure}
\begin{itemize}
    \item Single intermediary.
    \item Competing platforms could bid for high-$v$ products.
    \item Traffic-generating products may capture the rents.
\end{itemize}

\medskip

\item \textbf{Consumer behavior}
\begin{itemize}
    \item Search heterogeneity only.
    \item No taste heterogeneity or dynamic learning.
    \item Richer demand could change selection patterns.
\end{itemize}

\end{itemize}

\end{frame}


\end{document}