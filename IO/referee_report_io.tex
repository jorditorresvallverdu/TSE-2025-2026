\documentclass{article}
\usepackage{amsmath} 
\usepackage{amsfonts}
\usepackage{booktabs}
\usepackage[a4paper, margin=2.5cm]{geometry}
\usepackage{float}   % for [H]
\usepackage{graphicx}   % for \includegraphics
\usepackage{tabularx}
\usepackage[utf8]{inputenc}
\usepackage{geometry}
\usepackage{booktabs}
\usepackage{longtable}
\usepackage{blindtext}
\usepackage{hyperref}
\usepackage{natbib} % <-- NEW: to handle references
\usepackage{setspace}
\usepackage{array}
\usepackage{dcolumn}
\usepackage{threeparttable}
\usepackage{tikz}
\usepackage{amsmath}
\usetikzlibrary{decorations.pathreplacing}
\usepackage{pdflscape} % in your preamble
\usepackage{tabularray}
\setcounter{secnumdepth}{2}
\usepackage{amsmath, amsthm}  % for math and theorem environments
\setlength{\parskip}{0.45em}   % space between paragraphs
\setlength{\parindent}{0pt}    % optional: remove paragraph indentation

\usepackage{titlesec}

\titlespacing*{\section}
{0pt}      % left margin
{1.0em}    % space before section
{0.5em}    % space after section



\begin{document}

\title{Referee Report: Welfare effects of Buyer and Seller Power (Demirer, Rubens)}
\author{Jordi Torres}
\date{\today}


\maketitle


\section*{1. Introduction}
This paper studies how changes in buyer power affect welfare in vertical relationships when the type of vertical conduct is not observed \citep{DemirerRubens2025}. The welfare effects of buyer power depend on whether the market operates under monopsonistic or monopolistic conduct, but in many applications this distinction is not directly observable, which complicates policy analysis.

The authors propose a unified model in which both monopsonistic and monopolistic conduct are feasible given increasing upstream marginal costs and downward-sloping downstream demand. Unlike much of the IO and labor literature, vertical conduct is not imposed ex ante (it is not "assumed away") but selected endogenously through participation-based rules.

The main result is the existence of a unique efficient level of buyer power, $\beta^{*}$, that depends only on demand and cost elasticities. Comparing actual buyer power $\beta$ to this benchmark determines the implied conduct: monopsonistic when $\beta > \beta^{*}$ and monopolistic when $\beta < \beta^{*}$. 

The framework is applied to several empirical settings, including coal procurement in Texas, where the authors decompose observed distortions into monopoly and monopsony components and give policy advice on how should buyer power change to improve efficiency and consumer surpluss.

\section*{2. Model. Main Results}

Let $D$ denote the downstream firm and $U$ the upstream firm. The firms engage in bilateral Nash bargaining over a linear input price. The upstream firm faces increasing marginal costs, while the downstream firm faces a downward-sloping inverse demand function. Bargaining is static and both simultaneous and sequential timing are allowed. Under these assumptions, both monopsonistic and monopolistic vertical conduct may arise in equilibrium.

Formally, the equilibrium is characterized by the solution to
\[
\begin{cases}
\displaystyle \max_{q}\ \pi_i(w,q), 
& (1) \\[8pt]
\displaystyle 
\max_{w}\ 
\Big[ \pi_i(w,q) \Big]^{\beta}
\Big[ \pi_{i-1}(w,q) \Big]^{1-\beta}
\quad \text{s.t. } 
\pi_i \ge 0,\ \pi_{i-1} \ge 0,
& (2)
\end{cases}
\quad i \in \{\text{upstream},\ \text{downstream}\}.
\]

The contribution of the paper lies in defining vertical conduct in settings where it is not known ex-ante which party effectively controls quantities and where both monopsonistic and monopolistic conduct are feasible given the primitives of demand and costs. The key object governing this selection is the bargaining parameter $\beta$, which they interpret as buyer power (wrt 1-$\beta$, seller power).

First, the authors show that there exists a unique interior level of buyer power, $\beta^{*} \in (0,1)$, such that equilibrium output is maximized and coincides with the efficient bargaining outcome. This efficient level is given by
\[
\beta^{*} = \frac{-p'(q^{*})}{c'(q^{*}) - p'(q^{*})},
\]
and depends only on the relative elasticities of upstream marginal costs and downsteam demand. The intuition is similar to the Hosios\footnote{\citet{Hosios1990}} condition in the labor search literature: higher curvature of upstream costs requires greater seller power $(1-\beta)$ to counterbalance monopsony inefficiencies, while less elastic downstream demand requires greater buyer power $\beta$ to offset monopoly distortions.

The main theoretical result of the paper is given by Theorems 1 and 2. 
Under two alternative participation-based rules\footnote{
Vertical conduct is determined by simple participation conditions. 
Under the first, each firm participates in linear-price bargaining only if it earns a nonnegative markup or markdown. 
Under the second, quantity choice rights are given asymmetrically: under monopsonistic conduct, the upstream firm chooses output; while under monopolistic conduct the downstream firm does so. They do so provided that choosing quantity dominates bargaining over a linear contract. 
If neither party prefers to choose output, bargaining reverts to efficient two-part tariff contracting.
}, for any bargaining parameter $\beta \neq \beta^{*}$ there exists a unique equilibrium with trade in which exactly one type of vertical conduct arises. Specifically, if $\beta > \beta^{*}$, the equilibrium supports monopsonistic conduct, while if $\beta < \beta^{*}$, the equilibrium supports monopolistic conduct. Both conduct regimes cannot arise simultaneously, and when $\beta = \beta^{*}$ the equilibrium coincides with efficient bargaining, which also maximizes consumer surpluss.


\textbf{Limitations}
While elegant, the main limitation of the framework is that its policy implications are in partial equilibrium. Policy interventions are modeled as shifts in bargaining power $\beta$\footnote{That is, they implicitly assume that the effect of a merger in downstream firms only increases $\beta$.}, while the efficient benchmark $\beta^{*}$—which depends on demand and cost elasticities—is held fixed, even though both are equilibrium objects likely to change following mergers.

More crucially, the conduct-selection result relies on participation and incentive-compatibility constraints that are imposed under a very particular setting. As a result, the endogeneity of conduct is conditional on institutional frictions such as linear pricing, which limits the external validity of the results to settings where such frictions or institutional arrangements are not that empirically relevant. In this sense, it would be interesting to know if the results held under other institutional settings (non-linear contracts, quotas...etc). I understand this is better left as future research.


\section*{3. Empirical application}
The main empirical result of the paper is the application of Theorems 1 and 2 to decompose the source of vertical distortions using estimated primitives   (cost and demand) in a setting where vertical conduct is not directly observed.

The US coal industry is a natural setting for their framework. They show that contracts are mostly linear, upstream costs are increasing, downstream demand is downward sloping, and sufficient data are available to estimate the relevant primitives. This makes the empirical application consistent with the assumptions of the model.

Applying their framework to the Texas coal market, the authors find that 74.9\% of the observed distortion is due to double marginalization arising from monopoly power of coal mining firms, while the remaining share is attributed to monopsony power of electricity producers. Based on this decomposition, they conclude that, in this market, increases in buyer power are likely to be countervailing, while increases in seller power may be further distortionary.


\textbf{Limitations}
An important limitation, however, to the policy interpretation of these results (acknowledged by the authors-and related to the model critique) is that the efficient bargaining benchmark $\beta^{*}$ is itself an equilibrium object that depends on demand and cost elasticities. Policy interventions such as mergers are therefore likely to affect both the actual bargaining parameter $\beta$ and the efficient level $\beta^{*}$. As a result, conclusions based on changes in $\beta$ holding $\beta^{*}$ fixed should be interpreted as partial equilibrium. Studying specific policies would require jointly modeling how both $\beta$ and $\beta^{*}$ respond to changes in market structure.

A related limitation is that the empirical conclusions rely on a sharp threshold comparison between $\beta$ and $\beta^{*}$. Both objects are estimated with error, but uncertainty is not explicitly incorporated into the conduct classification or the distortion decomposition. Small measurement error in demand elasticities or bargaining parameters could lead to different conclusions about the source of distortions, especially when $\beta$ is close to $\beta^{*}$. This highlights the model dependence of the empirical exercise and raises concerns about robustness -also to functional forms of demand, costs- and external validity in other settings.

While external validity requires expanding the model's assumptions, the paper should provide some evidence of the dependence of the empirical conclusions to different functional forms and take into account uncertainty in estimation.



\section*{4. Conclusion and further research}

This paper contributes to the literature on vertical relations under monopsony and monopoly power by providing a unified framework in which both sources of market power can arise. Most of the existing literature imposes assumptions on cost or demand that effectively rule out one of these regimes ex ante, thereby fixing vertical conduct \citet{lamadon_mogstad, ho_kate,rubens_1,AlviarezEtAl2025,AzkarateAskasuaZerecero2025,CaldwellHaegeleHeining2025}. In contrast, this paper allows both monopsonistic and monopolistic conduct to be feasible given the primitives of the market.

The key theoretical contribution is to show that vertical distortions can be decomposed into monopsony and monopoly components using only demand and cost primitives. Under economically plausible -although justified only in limited settings- participation constraints, the authors characterize a unique efficient level of buyer power, $\beta^{*}$, at which output and consumer surpluss are maximized. Comparing actual bargaining power $\beta$ to this benchmark allows the researcher to infer the type of vertical conduct and the source of inefficiency when conduct is not directly observed. The empirical application to the Texas coal industry show how this decomposition can be implemented in practice.

Overall, the framework is best interpreted as a tool to decompose observed vertical distortions, rather than as a full policy evaluation framework for mergers or other interventions. A natural immediate direction for research would be to extend the model to allow for richer policy counterfactuals, for example by endogenizing $\beta^{*}$ and embedding the framework within a full merger simulation (mergers affect vertical relations beyond $\beta$, as is assumed in this paper). Also, as they argue in their paper, in future research it would be interesting to embed this framework in a dynamic setting, as vertical concerns are often related not only to static efficiency but also to investment and innovation incentives. 


\bibliographystyle{apalike}

\bibliography{bibliography}


\end{document}